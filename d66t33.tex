\documentclass[a4paper,english]{article}
\usepackage[T1]{fontenc}
\usepackage[latin9]{inputenc}
\setlength{\parskip}{\bigskipamount}
\setlength{\parindent}{0pt}
\usepackage{color}
\usepackage{babel}
\usepackage{amsmath}
\usepackage{amssymb}
\usepackage[unicode=true,
 bookmarks=false,
 breaklinks=false,pdfborder={0 0 1},backref=section,colorlinks=true]
 {hyperref}
\hypersetup{
 pdftex,linkcolor=blue,citecolor=blue,filecolor=blue,urlcolor=blue,pdftitle=,pdfauthor=,pdfsubject=,pdfkeywords=}
\usepackage{breakurl}

\makeatletter

\special{papersize=\the\paperwidth,\the\paperheight}

\usepackage{color}\usepackage{babel}




\pdfpageheight\paperheight
\pdfpagewidth\paperwidth

\usepackage{babel}
\usepackage{amsfonts}\usepackage{color}\usepackage{array}\usepackage{hhline}

% List styles
\newcounter{questenum}
\newcounter{techenum}
\newcommand{\liststyleQuestion}{%
\renewcommand\theenumi{\roman{enumi}}
\renewcommand\theenumii{\alph{enumii}}
\renewcommand\theenumiii{\arabic{enumiii}}
\renewcommand\theenumiv{\arabic{enumiv}}
\renewcommand\labelenumi{\theenumi.}
\renewcommand\labelenumii{\theenumii.}
\renewcommand\labelenumiii{\theenumiii.}
\renewcommand\labelenumiv{\theenumiv.}
}

\newenvironment{questionenum}
{ \liststyleQuestion
  \enumerate
  \setcounter{enumi}{\value{questenum}}
  \bfseries
  \itshape }
{ \endenumerate
  \setcounter{questenum}{\value{enumi}} }

\newcommand{\liststyleTechnicalQuestion}{%
\renewcommand\theenumi{\arabic{enumi}}
\renewcommand\theenumii{\alph{enumii}}
\renewcommand\theenumiii{\roman{enumiii}}
\renewcommand\theenumiv{\arabic{enumiv}}
\renewcommand\labelenumi{\theenumi.}
\renewcommand\labelenumii{\theenumii.}
\renewcommand\labelenumiii{\theenumiii.}
\renewcommand\labelenumiv{\theenumiv.}
}

\newenvironment{techitemize}
{ \liststyleTechnicalQuestion 
  \itshape
  \enumerate
  \setcounter{enumi}{\value{techenum}} }
{ \endenumerate
  \setcounter{techenum}{\value{enumi}} }
 

% Page layout (geometry)
\setlength{\voffset}{-1in}
\setlength{\hoffset}{-1in}
\setlength{\topmargin}{1.249cm}
\setlength{\oddsidemargin}{4.001cm}
\setlength{\textheight}{22.037998cm}
\setlength{\textwidth}{13.82cm}
\setlength{\footskip}{2.582cm}
\setlength{\headheight}{1.291cm}
\setlength{\headsep}{1.291cm}
% Footnote rule
\setlength{\skip\footins}{0.119cm}
\renewcommand{\footnoterule}{\vspace*{-0.018cm}\setlength\leftskip{0pt}\setlength\rightskip{0pt plus 
1fil}\noindent\textcolor{black}{\rule{0.25\columnwidth}{0.018cm}}\vspace*{0.101cm}}
% Pages styles

\newcommand{\ps@Standard}{
  \renewcommand\@oddhead{D6.6 / Pilots in property based testing, Pilot report \#6}
  \renewcommand\@evenhead{\@oddhead}
  \renewcommand\@oddfoot{\thepage{}}
  \renewcommand\@evenfoot{\@oddfoot}
  \renewcommand\thepage{\arabic{page}}
}

\title{}
\author{}
\date{2015-09-01}

\makeatother


\begin{document}

\setcounter{page}{1}\pagestyle{Standard} \textbf{Task 3.3: Tools and techniques to model the differences between 
versions of a system that are parametrized or configured in different ways.}

\textit{Status: planned}

This pilot study will focus on open source software, namely the Erlang/OTP \textbf{Dets} system, that provides 
disk-based term storage case, and which underlies the mnesia Erlang database. The stored terms (called
objects here) are tuples in which one element is defined to be the key, and a Dets table is a collection of objects 
with the key at the same position stored on a file.

Dets provides an appropriate case study for three reasons:

\begin{itemize}
\item There are three types of Dets tables: set, bag and duplicate\_bag, and these can be seen as significantly 
different variants of a single parametrised model.
\item There are two versions of the format used for storing objects on file are supported by Dets. The first version, 
8, is the format always used for tables created by OTP R7 and earlier; the second version, 9, is the default version of 
tables created by OTP R8 (and later OTP releases), but OTP R8 can create version 8 tables, and convert version
8 tables to version 9, and \textit{vice versa}.
\item Dets tables can also be compared with Ets tables, that provide an in memory variant of term storage. 
\end{itemize}

\begin{questionenum}
\item What will you try to find out with this study
\end{questionenum}


The research questions we aim to answer in this pilot study are:

\textit{A: Is it technically feasible to construct parametric models of Dets tables that}

\begin{techitemize}
\item express the difference between set, bag and duplicate\_bag variants?
\item express the difference between successive versions of the Dets implementation?
\item express the difference between Dets and Ets tables?
\end{techitemize}

\textit{B: Is the process of devising these parametric models worthwhile from a business point of view: can the process 
be accomplished at a cost appropriate to the added value that they provide?}

\textit{C: Do the parametric models effectively augment the existing models? This can be shown by discovering new 
faults in the Dets system,or by uncovering ``past'' faults in older versions of the system?}

\textit{D: Is the method assessed as accurate, quality enhancing and useful by the assessors of the pilot study?}

\begin{questionenum}
\item Who will execute this study
\end{questionenum}

This pilot will be executed by staff at Kent. Other consortium members will be asked to assess the results of the work 
done by the Kent staff.

\begin{questionenum}
\item What data will you measure in order to get some answers to the things you want to find out.
\end{questionenum}

A: These questions will be answered through the construction of the models, and through an assessment of this process 
by the assessors of the pilot study.

B: In order to answer these questions, participants in the study will keep an accurate record of the time taken to 
develop the models. Once complete, the other measurements will be made on the completed code.

C: The data here will be generated through the corresponding testing activity.

D: These questions will be answered by asking assessors to complete an appropriate questionnaire.

\begin{questionenum}
\item What will be the steps to execute this study and collect the data
\end{questionenum}

The initial work will be undertaken by staff from Kent, and its results will be assessed by staff from another partner 
in the consortium.

\begin{questionenum}
\item What were the results?
\end{questionenum}

Yet to be done.

\begin{questionenum}
\item Threats to validity
\end{questionenum}

\begin{itemize}
\item Data points are low. 
\item Users have high domain knowledge not necessarily representative of an average user. 
\item Specialised nature of the particular models considered. 
\end{itemize}

\clearpage

\textbf{Bibliography and references}

\end{document}
